\section{Related Work}

Since the early days of CTs health manufacturer were attempted to reduce
radiation exposure in CT scans by using i.e. more sensible detection
electronics and more sophisticated scanning sequences. Through the growing
availability of computing power we also find evermore computer vision
techniques being utilized, for example to enhance the image quality of
low-dose CTs \cite{Xu12}. Altough these efforts have lead to an
impressive and steady evolution of CT apparatus, they still require the
patient to be irradiated.

First approaches which dispense with radiation exposure, relay on the
atlas based transformations applied to MR to predict CT \cite{Hofmann08}.
Further improvements included i.e. random forests \cite{Andreasen13}.
It has been shown that CT prediction can in fact replace CT for treatment
planning \cite{Andreasen2017}.

In computer science we have seen an incredible progress with
deep learning techniques in a variety of areas including natural language
processing and computer vision \cite{LeCun15}. Recent efforts with generative
adversarial networks (GANs) \cite{Goodfellow14} seem to be a promising
path towards the challenge of finding appropriate target functions to optimize
through the use of game theory. Successful applications in computer vision
in which GANs proved significant improvements over the former state of art
include the task of image to image translation \cite{Isola16} but also the
generalization of three dimensional structures inside the so called latent
space \cite{ZXFT16}.

This in mind the medical computer vision community rapidly adapted
GANs for their own specific tasks which in comparison to computer
vision typical involve volumetric single channel images with high bit depth.
Bearing the challenge of CT from MR prediction the expectations towards GANs
have been met showing overall better results compared to the previous
approach \cite{Nie16}. Additionally the full potential of GANs have not
been exhausted yet as it also has been shown that GANs are capable of
being trained with unregistered modalities \cite{Wolterink17}.

Beside the enourmous breakthroughs made in medical computer vision we still
see a shortage in a reproducable comparison of recent methods with publicly
available data. Not to mention the open questions with regard to best
practices in choosing good GAN model parameters for the task of CT prediction
which we hope to address in the subsequent sections.

