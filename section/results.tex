\section{Experiments}

We had difficulties with the implementation of the patch aggregation required
for the implementation of the context-aware 3D synthesis. Though patch
aggregation worked in general, it occupied more computational resources than
the actual training. Unfortunatly we were not able to find a workaround or to
allocate more computation quota. Therefor we did not perform more than one
iteration of the auto-context model, which prevented a fair comparison.
However, we encourage everyone to test our implementation themselves.

As a consequence we only conducted experiments with the u-net \gls{cnn} and
the pixtopix \gls{gan}.

\subsection{u-net}

The \num{17} subjects of the dataset were divided into \gls{13} subjects for
training and \gls{4} subjects for validation. We tried to respect the
transverse resolution of the initial volumes, i.e.\ the number of transverse
slices, in the division process. In \Cref{tab:unet:dataset} we see the volume
shape of the input \gls{mri} and the target \gls{ct} of the respective
subject as well as their assignment to the training or validation dataset.
\begin{table}[h]
  \centering
  \begin{tabular}{ccc}
    \toprule
    Dataset & Subject & Shape \\
    \midrule
    Training & \num{1} & \num{161x320x250x1} \\
    Training & \num{2} & \num{149x328x250x1} \\
    Training & \num{3} & \num{112x303x281x1} \\
    Training & \num{4} & \num{155x291x259x1} \\
    Training & \num{5} & \num{143x307x284x1} \\
    Training & \num{6} & \num{149x278x267x1} \\
    Training & \num{7} & \num{200x289x268x1} \\
    Training & \num{8} & \num{218x282x238x1} \\
    Training & \num{9} & \num{191x322x252x1} \\
    Training & \num{10} & \num{200x303x243x1} \\
    Training & \num{11} & \num{181x317x239x1} \\
    Training & \num{12} & \num{186x310x248x1} \\
    Training & \num{13} & \num{112x313x238x1} \\
    Validation & \num{1} & \num{112x298x227x1} \\
    Validation & \num{2} & \num{223x328x282x1} \\
    Validation & \num{3} & \num{223x307x276x1} \\
    Validation & \num{4} & \num{204x329x262x1} \\
    \bottomrule
  \end{tabular}
  \caption{Training and validation dataset volumes used in this section. The
    dimensions of the shape correspond to depth, height and width.
  }\label{tab:unet:dataset}
\end{table}
The subjects assigned to the training dataset were processed in transverse
slices. We trained the u-net model once with the \gls{mae} and once with the
\gls{mae} and \gls{gdl} loss in order to estimate the impact of the \gls{gdl}.
The training parameters are summarized in \Cref{tab:unet:params}. The image
slices correspond to the total number of 2D images extracted from the
transverse (depth) plane of the volumes. The batch size denotes the number of
images proccessed in one step. For convenience we estimated the number of
epochs from the total training step number. The training was stopped when the
gradients vanished and the total loss stabilized. We found that these criteria
were met for the u-net model at around \num{20000} steps or \num{140} epochs.
\begin{table}[h]
  \centering
  \begin{tabular}{cccccc}
    \toprule
    & &
    \multicolumn{2}{c}{\acrshort{mae}} &
    \multicolumn{2}{c}{\acrshort{mae}+\acrshort{gdl}} \\
    Image Slices & Batch Size & Steps & Epochs & Steps & Epochs \\
    \midrule
    \num{2157} & \num{16} & \num{20542} & \num{152} & \num{19388} & \num{143} \\
    \bottomrule
  \end{tabular}
  \caption{Training parameters used for the distance metrics experiments.
  }\label{tab:unet:params}
\end{table}
The appropriate loss term weights $\lambda_\text{mae}$ and
$\lambda_\text{mse}$ were chosen such that the gradient with respect to the
loss terms yields a similar magnitude. We found that to be the case for
\num{1e-7}. In an early attempt we also tried to compare \gls{mae} and
\gls{mse} as loss functions, yet, we did not find significant differences and
stuck with \gls{mae} which is the distance loss used in the original pixtopix. 
\begin{table}[h]
  \centering
  \begin{tabular}{ccccc}
    \toprule
    $\lambda_\text{mae}$ &
    $\lambda_\text{gdl}$ &
    \acrshort{mae} &
    \acrshort{mse} &
    \acrshort{psnr} \\
    \midrule
    \num{1} & \num{0} & \num{31.58} & \num{6577} & \num{59.5} \\
    \num{1} & \num{1e-7} & \num{37.15} & \num{7945} & \num{58.1} \\
    \bottomrule
  \end{tabular}
  \caption{Distance metrics for the u-net model trained with different loss
    functions, evaluated on the training dataset.
  }\label{tab:unet:training}
\end{table}
\Cref{tab:unet:training} lists the metrics for the u-net model trained with
different loss functions evaluated on the training dataset. The \gls{psnr}
metric is defined as
\begin{equation}
	\psnr{\left(X,Y\right)}
	=
	10\log{\left(\frac{M^2}{\mse}\right)}
	\label{eq:psnr},
\end{equation}
wherein $M$ denotes the maximum pixel value, in our case $2^{16}-1$. It is
useful to quantify the noise level present in the generated \gls{ct} images,
with a higher \gls{psnr} usually corresponding to lower noise. We remark
that the \gls{gdl} yields a slightly better result on the \gls{psnr} metric,
but yields inferior values on \gls{mae} and \gls{mse}.
\begin{table}[h]
  \centering
  \begin{tabular}{ccccc}
    \toprule
    $\lambda_\text{mae}$ &
    $\lambda_\text{gdl}$ &
    \acrshort{mae} &
    \acrshort{mse} &
    \acrshort{psnr} \\
    \midrule
    \num{1} & \num{0} & \num{123.6} & \num{70846} & \num{47.93} \\
    \num{1} & \num{1e-7} & \num{129.0} & \num{72704} & \num{47.90} \\
    \bottomrule
  \end{tabular}
  \caption{Distance metrics for the u-net model trained with different loss
    functions, evaluated on the validation dataset.
  }\label{tab:unet:validation}
\end{table}
In \Cref{tab:unet:validation} we see the same metrics evaluated on the
validation dataset. These metrics are in general more informative than the
metrics from the training dataset as we expect the networks to overfit. In
comparison to the \Cref{tab:unet:training} the \gls{mae} are nearly four times
the \gls{mae} for the training dataset. The \gls{mse} is of one magntiude
higher which also confirms overfitting. The \gls{psnr} metric obtained from
the validation dataset is lower than for the training dataset. We should keep
in mind that for the \gls{psnr} a higher value is usually better and also that
the \gls{psnr} is scaled logarithmicly. Overall the metrics suggest that our
network overfits and that the \gls{gdl} slightly decreases the performance.
Nevertheless we should keep in mind that these metrics are based on
pixel-wise measures, therefore we need to examine the visual results to draw
final conclusions and sort out, for instance, possible bias in the subject
selection of the datasets.
\begin{figure}[h]
  \centering
  \begin{adjustbox}{width=\linewidth}
    \inputpgf{figure}{unet-training.pgf}
  \end{adjustbox}
  \caption{Transverse views for the u-net model trained with different
    loss functions, evaluated on the training dataset.
  }\label{fig:unet:training}
\end{figure}
In \Cref{fig:unet:training} we present the the transverse views for the
differently trained u-net models evaluated on the training dataset including
the ground truth on the left. We note that the u-net instance trained with
\gls{gdl} shows some artifacts outside of the head. Furthermore the soft
matter structure seems more coarse.
\begin{figure}[h]
  \centering
  \begin{adjustbox}{width=\linewidth}
    \inputpgf{figure}{unet-validation.pgf}
  \end{adjustbox}
  \caption{Transverse views for the u-net model trained with different
    loss functions, evaluated on the validation dataset.
  }\label{fig:unet:validation}
\end{figure}
In \Cref{fig:unet:validation} we show the transverse views for the differently
trained u-net models evaluated on the validation dataset. We can see that
the overall performance is much worse to unknown data which again suggests
overfitting. For the first two rows we note that the u-net trained with
\gls{gdl} loss seems more robust. We conclude that the \gls{gdl} improves
subjective performance on the validation dataset by a small amount, but
performance by standard metric seems to be decreased by a small amount.
Furthermore we want to mention, that the use of the \gls{gdl} requires high
computational cost.

\subsection{pixtopix}

In a second part we want to compare the u-net model trained on the \gls{mae}
with the pixtopix model. As already noted both models differ in that the
pixtopix uses a discriminator network in order to calculate a least-square
adverarial loss term. The adversarial least-square loss term was waited with
$\lambda_\text{adv}=0.01$ and the \gls{mae} term with $\lambda_\text{mae}=1$.

Additionaly we manualy removed incomplete slices from the dataset. These
incomplete slices arise from the coregistration routine when one volume is
tilted but does not cover the same region of the fixed volume because of
different resolution. In \Cref{tab:unet_pixtopix:dataset} we present the
volumes shapes of the cleaned dataset. In comparison to
\Cref{tab:unet:dataset} the transverse (depth) resolution has been reduced
by the incomplete transverse slices we removed.
\begin{table}[h]
  \centering
  \begin{tabular}{ccc}
    \toprule
    Dataset & Subject & Shape \\
    \midrule
    Training & \num{1} & \num{137x320x250x1} \\
    Training & \num{2} & \num{130x328x250x1} \\
    Training & \num{3} & \num{111x303x281x1} \\
    Training & \num{4} & \num{143x291x259x1} \\
    Training & \num{5} & \num{141x307x284x1} \\
    Training & \num{6} & \num{148x278x267x1} \\
    Training & \num{7} & \num{198x289x268x1} \\
    Training & \num{8} & \num{208x282x238x1} \\
    Training & \num{9} & \num{162x322x252x1} \\
    Training & \num{10} & \num{185x303x243x1} \\
    Training & \num{11} & \num{180x317x239x1} \\
    Training & \num{12} & \num{184x310x248x1} \\
    Training & \num{13} & \num{93x313x238x1} \\
    Validation & \num{1} & \num{105x298x227x1} \\
    Validation & \num{2} & \num{190x328x282x1} \\
    Validation & \num{3} & \num{202x307x276x1} \\
    Validation & \num{4} & \num{198x329x262x1} \\
    \bottomrule
  \end{tabular}
  \caption{Training and validation dataset volumes used in this section. The
    dimensions of the shape correspond to depth, height and width.
  }\label{tab:unet_pixtopix:dataset}
\end{table}
\Cref{tab:unet_pixtopix:params} lists the training parameters used in the
following experiments. The pixtopix model required more training steps to
converge.
\begin{table}[h]
  \centering
  \begin{tabular}{cccccc}
    \toprule
    & &
    \multicolumn{2}{c}{u-net (\gls{mae})} &
    \multicolumn{2}{c}{pixtopix} \\
    Image Slices & Batch Size & Steps & Epochs & Steps & Epochs \\
    \midrule
    \num{2020} & \num{16} & \num{22080} & \num{174} & \num{37832} & \num{299} \\
    \bottomrule
  \end{tabular}
  \caption{Training parameters used for the u-net and pixtopix comparison.
  }\label{tab:unet_pixtopix:params}
\end{table}
\Cref{tab:unet_pixtopix:training} summarizes the evaluation metrics obtained
for the training dataset. Comparison to \Cref{tab:unet:training} has to be
done carefully as we used differently preprocessed datasets.
\begin{table}[h]
  \centering
  \begin{tabular}{ccccc}
    \toprule
    Model & Loss &
    \acrshort{mae} &
    \acrshort{mse} &
    \acrshort{psnr} \\
    \midrule
    u-net & \acrshort{mae} & \num{90.5} & \num{61853} & \num{49.4} \\
    pixtopix & least-square & \num{21.6} & \num{4210} & \num{60.4} \\
    \bottomrule
  \end{tabular}
  \caption{Distance metrics for the u-net model trained with \acrshort{mae}
    loss compared with the pixtopix model trained with least-square adversarial
    loss, evaluated on the training dataset.
  }\label{tab:unet_pixtopix:training}
\end{table}
\Cref{tab:unet_pixtopix:validation} lists the evaluation metrics obtained from
the validation dataset. Compared to \Cref{tab:unet_pixtopix:training} the
u-net metrics differ not as much in our former experiments with only the
u-net architecture. Furthermore we see a large decrease of the pixtopix
performance on the validation dataset.
\begin{table}[h]
  \centering
  \begin{tabular}{ccccc}
    \toprule
    Model & Loss &
    \acrshort{mae} &
    \acrshort{mse} &
    \acrshort{psnr} \\
    \midrule
    u-net & \acrshort{mae} & \num{136.9} & \num{101943} & \num{46.77} \\
    pixtopix & least-square & \num{112.7} & \num{82173} & \num{47.55} \\
    \bottomrule
  \end{tabular}
  \caption{Distance metrics for the u-net model trained with \acrshort{mae}
    loss compared with the pixtopix model trained with least-square adversarial
    loss, evaluated on the validation dataset.
  }\label{tab:unet_pixtopix:training}
\end{table}
The visual comparison of the u-net and pixtopix model on the training dataset,
see \Cref{fig:unet_pixtopix:training} shows very good results for both models.
The pixtopix model, however, does not show artifacts. Overall the hard and
soft matter tissue look very similar to the ground truth.
\begin{figure}[h]
  \centering
  \begin{adjustbox}{width=\linewidth}
    \inputpgf{figure}{unet-pixtopix-training.pgf}
  \end{adjustbox}
  \caption{Transverse views for the u-net model trained with \acrshort{mae}
    loss compared with the pixtopix model trained with least-square adversarial
    loss, evaluated on the training dataset.
  }\label{fig:unet_pixtopix:training}
\end{figure}
The visual comparison of the u-net and pixtopix model on the validation
dataset, see \Cref{fig:unet_pixtopix:validation} shows that even though the
metrics on the validation dataset decreased much more relative to the u-net
metrics, the visual results are much better. For ananatomical characteristics
general to the human head the pixtopix model is able to successfully reproduce
\gls{ct} representation from \gls{mri}, however for anatomical features that
differ greatly between subjects, results are not good.
\begin{figure}[h]
  \centering
  \begin{adjustbox}{width=\linewidth}
    \inputpgf{figure}{unet-pixtopix-validation.pgf}
  \end{adjustbox}
  \caption{Transverse views for the u-net model trained with \acrshort{mae}
    loss compared with the pixtopix model trained with least-square adversarial
    loss, evaluated on the validation dataset.
  }\label{fig:unet_pixtopix:validation}
\end{figure}
Overall we can confirm a large improvement of the \gls{gan} approach compared
with \gls{cnn} approach. Altough both architecture use the same network for
the prediction, the adversarial loss term in pixtopix is able to guide the
optimizer to a better local extrema.

\subsection{Gradient Boost}

As a third experiment we wanted to improve the soft tissue structure of the
synthesized \gls{ct}s. Therefore we used skull extraction to create masks of
the brain volume. These masks were then used to increase gradient weight in
fine-tuning the pixtopix model.
\begin{table}[h]
  \centering
  \begin{tabular}{cccccc}
    \toprule
    & &
    \multicolumn{2}{c}{u-net} &
    \multicolumn{2}{c}{pixtopix} \\
    Image Slices & Batch Size & Steps & Epochs & Steps & Epochs \\
    \midrule
    \num{2020} & \num{16} & \num{37832} & \num{299} & \num{51911} & \num{411} \\
    \bottomrule
  \end{tabular}
  \caption{Training parameters used for pixtopix and gradient boosted
    fine-tuned pixtopix model.
  }\label{tab:pixtopix:params}
\end{table}
In \Cref{tab:pixtopix:params} we summarized the training parameters for the
fine-tuning experiment. The untuned pixtopix, described in the prevous section,
was trained for about \num{299} epochs. Then we amended the gradient
calculation to increase weight of the soft-tissue area and proceeded to train
for about \num{112} more epochs.
\begin{table}[h]
  \centering
  \begin{tabular}{cccc}
    \toprule
    Model & \acrshort{mae} & \acrshort{mse} & \acrshort{psnr} \\
    \midrule
    pixtopix & \num{21.56} & \num{4210} & \num{60.38} \\
    pixtopix (fine-tuned) & \num{23.37} & \num{4140} & \num{60.30} \\
    \bottomrule
  \end{tabular}
  \caption{Distance metrics for the pixtopix model trained with least-square
    adversarial loss compared to fine-tuned with gradient boost, evaluated on
    the training dataset.
  }\label{tab:pixtopix:training}
\end{table}
\begin{table}[h]
  \centering
  \begin{tabular}{cccc}
    \toprule
    Model & \acrshort{mae} & \acrshort{mse} & \acrshort{psnr} \\
    \midrule
    pixtopix & \num{112.73} & \num{82173} & \num{47.55} \\
    pixtopix (fine-tuned) & \num{112.23} & \num{79296} & \num{47.68} \\
    \bottomrule
  \end{tabular}
  \caption{Distance metrics for the pixtopix model trained with least-square
    adversarial loss compared to fine-tuned with gradient boost, evaluated on
    the validation dataset.
  }\label{tab:pixtopix:validation}
\end{table}
In \Cref{tab:pixtopix:training} and \Cref{tab:pixtopix:validation} the
evaluation metrics on the training and validation dataset comparing the
pixtopix model with the fine-tuned pixtopix model are presented. For the
\gls{mae} and \gls{mse} we see a small improvement of the fine-tuned model
on the validation dataset.
\begin{figure}[h]
  \centering
  \begin{adjustbox}{width=\linewidth}
    \inputpgf{figure}{pixtopix-training.pgf}
  \end{adjustbox}
  \caption{Transverse views for the pixtopix model trained with least-square
    adversarial loss compared to fine-tuned with gradient boost, evaluated on
    the validation dataset.
  }\label{fig:pixtopix:training}
\end{figure}
\begin{figure}[h]
  \centering
  \begin{adjustbox}{width=\linewidth}
    \inputpgf{figure}{pixtopix-validation.pgf}
  \end{adjustbox}
  \caption{Transverse views for the pixtopix model trained with least-square
    adversarial loss compared to fine-tuned with gradient boost, evaluated on
    the validation dataset.
  }\label{fig:pixtopix:validation}
\end{figure}
In \Cref{fig:pixtopix:training} and \Cref{fig:pixtopix:validation} we show
the visual results of the two pixtopix variants. Altough some results, for
instance the last row from the training results, suggest an improved fine
structure of the soft tissue, we cannot definetly conclude that fine-tuning
with incresed soft-tissue weights yields better soft-tissue results, however,
we should keep in mind that we might not found the correct fine-tuning
parameters and that a further increase in order to compensate for the
exponential decay of the optimizer is necessary.
