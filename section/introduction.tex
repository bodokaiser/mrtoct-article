\section{Introduction}

\gls{ct} and \gls{mri} are the essential medical imaging modalities for
clinical diagnosis and cancer monitoring. Inside the clinical framework
\gls{mri} is the more informative and safer modality~\cite{Hartwig09}. Instead
of x-rays which are known to contribute to carcinogenesis~\cite{Martin06},
\gls{mri} exploits the magnetic properties of the hydrogen nucleus and is not
associated with to have negative impact on the patients health. In addition
\gls{mri} provides more detailed visual information on soft tissue. These
benefitial characteristics suggest that \gls{mri} supersedes \gls{ct} in the
long-term. One of many remaining obstacles is, however, the requirement of
\gls{ct} for image guided radiation therapy planning. Although \gls{mri} and
\gls{ct} differ significant in the applied physics, the high entropy of
\gls{mri} data suggests the existence of a surjective transform from
\gls{mri} to \gls{ct} space. With the recent advents in computer vision
techniques based on \gls{gan} we seem to close to finding such a mapping
emprically.
