\section{Related Work}

Since the early days of \gls{ct}, health manufacturer were attempted to reduce
radiation exposure in \gls{ct} scans by using, for instance, more sensible
detection electronics, and more sophisticated scanning sequences. Through the
growing availability of computing power we also find evermore computer vision
techniques being utilized, for example, in the enhancement of image quality of
low-dose \gls{ct}s~\cite{Xu12}. Altough these efforts have lead to an
impressive and steady evolution of the \gls{ct} apparatus, they still require
the patient to be irradiated nevertheless.
First approaches which dispense the radiation exposure, through the
computational transformation of \gls{mri} to \gls{ct}, relay on the
atlas-based transformations applied to \gls{mri} to predict \gls{ct}, see
Ref.~\cite{Hofmann08}. Further improvements thereto include, for instance,
random forests~\cite{Andreasen13}. Finally it has been shown that these
\gls{ct} prediction methods can in fact already replace physical \gls{ct}
for treatment planning~\cite{Andreasen2017}.
At the same time, we have seen an incredible progress with deep learning
techniques in computer science~\cite{LeCun15}. Recent efforts with \gls{gan}s,
see Ref.~\cite{Goodfellow14}, seem to be a promising path towards finding
a global optimum in training neural networks through the use of game theory.
Furthermore \gls{gan}s proved significant improvements over the former state
of art in the task of image to image translation~\cite{Isola16} but also the
generalization of three dimensional structures inside the so called latent
space~\cite{ZXFT16}.
Keeping this in mind, the medical computer vision community rapidly adapted
\gls{gan}s for their own specific tasks. In comparison to datasets common in
general computer vision, medical datasets typical comprise volumetric single
channel images with high bit depth. Bearing the challenge of \gls{ct} from
\gls{mri} prediction in mind, the expectations towards \gls{gan}s have been
lately shown increased performance to the previous approaches~\cite{Nie16}.
Yet, the full potential of \gls{gan}s have not been exhausted. For example,
it has been shown that \gls{gan}s are capable of being trained with
unregistered modalities~\cite{Wolterink17}.
Beside the enourmous breakthroughs made in medical computer vision we still
see a shortage in a reproducable comparison of recent methods with publicly
available data. Not to mention the open questions with regard to best
practices in choosing good \gls{gan} model parameters for the task of
\gls{ct} prediction which we hope to address in the subsequent sections.
