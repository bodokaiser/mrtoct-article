\section{Introduction}

Computer tomography (CT) and magnetic resonance imaging (MR) are the main
workhorses of clinical diagnosis and cancer monitoring as they reveal the
condition of the patients inner organs.

In consequence of CT and MR exploiting distinct physical effects both
modalities take complementary roles inside the present clinical framework
with MR being more the informative and safe~\cite{Hartwig09} but also
more expensive modality.


While CT is based on the interaction of high energy photons (x-rays) with
biological tissue, MR 


, therefore so called image guided planning based on the
patients CT scan is used to estimate the patients specific radiation treatment
planning. Even though x-rays used for imaging are of a lower energy band
as x-rays used for cancer treatment, it has been shown that they still possess
a risk of developing new cancer inside the patient~\cite{Martin06}.
MR on the other hand takes advantage of the magnetic properties of hydrogen
and is not associated with any health risks . Furthermore MR
provides a much higher soft tissue contrast which is useful for cancer
classification. Although MR and CT differ significant in the applied
physics the high entropy in MR data suggests the existence of a one
directional mapping from MR to CT space whereby the acquisition of CT would
become obsolete. Beside the stated health benefits for the patients such an
approach would reduce expanses and also would free CT resources for
emergency cases where the fast acquisition of CT is used to locate internal
bleedings.
